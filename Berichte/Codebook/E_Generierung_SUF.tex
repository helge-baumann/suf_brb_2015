\section{Vom Rohdatensatz zum Scientific Use File}



\subsection{Missing Labels}\label{var_miss}

Die \glqq ungültigen\grqq\xspace bzw. fehlenden Werte (\textit{missing values}) des Rohdatensatzes wurden zwecks Nutzerfreundlichkeit überarbeitet. Zum einen wurden die bereits gelabelten Werte für \glqq Weiß nicht\grqq , \glqq verweigert\grqq\xspace und \glqq trifft nicht zu\grqq\xspace auf negative Werte gesetzt, um Berechnungen zu vereinfachen. Zum anderen wurden ungelabelte, fehlende Werte dahingehend differenziert, ob sie durch die \hyperref[langkurz]{Fragebogenversion} oder eine etwaige Filterführung fehlen (die Filterführungen können durch den Fragebogen nachvollzogen werden). Folgende Transformationen wurden vorgenommen:

\begin{itemize}
	\item \glqq Weiß nicht\grqq -Angaben:
	\begin{itemize}
	\item im Rohdatensatz 8, 98, 998, 9998, 999998, 999998 
	\item im Scientific Use File durchgehend -8
	\end{itemize}
	\item Verweigerungen:
	\begin{itemize}
	\item im Rohdatensatz 7, 97, 997, 9997, 999997, 999997 
	\item im Scientific Use File durchgehend -7
	\end{itemize}
	\item \glqq Trifft nicht zu\grqq -Angaben:
	\begin{itemize}
	\item im Rohdatensatz 6
	\item im Scientific Use File durchgehend -6
	\end{itemize}
	\item \glqq Missing durch Filterführung\grqq : -5
	\item \glqq Item in Fragebogenversion nicht erhoben\grqq : -4

\end{itemize}

\subsection{Generierte Variablen}\label{var_generiert} 

Im Folgenden wird der Erzeugungsprozess generierter Variablen beschrieben. Dabei kann im Einzelnen
nachvollzogen werden, durch welche Ursprungsvariablen bzw. welche Werte dieser Ursprungsvariablen die neuen, generierten
Variablen zustande kommen. Generiert wurden solche Variablen, die sich aus mehreren Fragen zusammensetzen. Dies betrifft einerseits Anteilswerte von Beschäftigtenstrukturangaben, die aus der Division von  der jeweiligen Beschäftigtenstruktur und der Betriebsgröße (Variable \hyperref[var:D1]{D1}) generiert wird. Dies betrifft andererseits die Struktur des Betriebsrates, für die die jeweils erhobenen Angaben (z.B., wie viele Frauen es im Betriebsrat gibt) durch die Größe des Betriebsrates (Variable \hyperref[var:M3]{M3}) dividiert wurden. Diese Variablen sind im Datensatz des Scientific Use Files durch das Suffix \glqq \_gen \grqq\xspace gekennzeichnet, auch ihr Variablenlabel enthält einen entsprechenden Hinweis (\grqq generierte Variable\grqq )

\begin{longtable}{!{\color{black}\vline width 1pt} L{2.5cm} !{\color{black}\vline width 1pt} L{9.125cm} | L{5.125cm} !{\color{black}\vline width 1pt}  }
	
	\toprule
		\textbf{Variable} & \textbf{WENN:} & \textbf{DANN:}  \\ 
	\midrule
	\endfirsthead
	
	\toprule
		\textbf{Variable} & \textbf{WENN:} & \textbf{DANN:}  \\ 
	\midrule
	\endhead
	
	\midrule
	
	\endfoot
	\bottomrule
	\endlastfoot
		
	% latex table generated in R 3.6.3 by xtable 1.8-4 package
% Tue Mar 17 12:15:15 2020
 \textbf{C1\_1\_gen}\phantomsection\label{C1:1:gen} & \hyperref[var:C1]{C1} $  <  $ 999996 \& \hyperref[var:D1]{D1} $ < $ 999997 & C1\_1\_gen = $ \frac{\hyperref[var:C1]{C1}}{\hyperref[var:D1]{D1}} $ \\ 
   & (\hyperref[var:C1]{C1} = " weiß nicht" \xspace ODER \hyperref[var:D1]{D1} = "weiß nicht") \& \hyperref[var:C1:1]{C1\_1} $ \neq $ "verweigert" \& \hyperref[var:C1:1]{C1\_1} $ \neq $ \grqq Angabe in Prozent" \& \hyperref[var:D1]{D1} $ \neq $ "verweigert"  & C1\_1\_gen = -8 \\ 
   & (\hyperref[var:C1:1]{C1\_1} = "verweigert" \xspace ODER \hyperref[var:D1]{D1} = "verweigert") \& \hyperref[var:C1:1]{C1\_1} $ \neq $ \grqq Angabe in Prozent"  & C1\_1\_gen = -7 \\ 
   & \hyperref[var:C1:1]{C1\_1} = \grqq Angabe in Prozent"  & C1\_1\_gen = \hyperref[var:C1:1]{C1\_1} \\ 
   \midrule
\textbf{D1aa\_gen}\phantomsection\label{D1aa:gen} & \hyperref[var:D1a]{D1a} $  <  $ 999996 \& \hyperref[var:D1]{D1} $ < $ 999997 & D1aa\_gen = $ \frac{\hyperref[var:D1a]{D1a}}{\hyperref[var:D1]{D1}} $ \\ 
   & (\hyperref[var:D1a]{D1a} = " weiß nicht" \xspace ODER \hyperref[var:D1]{D1} = "weiß nicht") \& \hyperref[var:D1aa]{D1aa} $ \neq $ "verweigert" \& \hyperref[var:D1aa]{D1aa} $ \neq $ \grqq Angabe in Prozent" \& \hyperref[var:D1]{D1} $ \neq $ "verweigert"  & D1aa\_gen = -8 \\ 
   & (\hyperref[var:D1aa]{D1aa} = "verweigert" \xspace ODER \hyperref[var:D1]{D1} = "verweigert") \& \hyperref[var:D1aa]{D1aa} $ \neq $ \grqq Angabe in Prozent"  & D1aa\_gen = -7 \\ 
   & \hyperref[var:D1aa]{D1aa} = \grqq Angabe in Prozent"  & D1aa\_gen = \hyperref[var:D1aa]{D1aa} \\ 
   \midrule
\textbf{D1bb\_1\_1\_gen}\phantomsection\label{D1bb:1:1:gen} & \hyperref[var:D1b:1]{D1b\_1} = "nein" & \hyperref[var:D1bb:1:1:gen]{D1bb\_1\_1\_gen} = 0 \\ 
   & \hyperref[var:D1b:1:1]{D1b\_1\_1} $  <  $ 999996 \& \hyperref[var:D1]{D1} $ < $ 999997 & D1bb\_1\_1\_gen = $ \frac{\hyperref[var:D1b:1:1]{D1b\_1\_1}}{\hyperref[var:D1]{D1}} $ \\ 
   & (\hyperref[var:D1b:1:1]{D1b\_1\_1} = " weiß nicht" \xspace ODER \hyperref[var:D1]{D1} = "weiß nicht") \& \hyperref[var:D1bb:1:1]{D1bb\_1\_1} $ \neq $ "verweigert" \& \hyperref[var:D1bb:1:1]{D1bb\_1\_1} $ \neq $ \grqq Angabe in Prozent" \& \hyperref[var:D1]{D1} $ \neq $ "verweigert"  & D1bb\_1\_1\_gen = -8 \\ 
   & (\hyperref[var:D1bb:1:1]{D1bb\_1\_1} = "verweigert" \xspace ODER \hyperref[var:D1]{D1} = "verweigert") \& \hyperref[var:D1bb:1:1]{D1bb\_1\_1} $ \neq $ \grqq Angabe in Prozent"  & D1bb\_1\_1\_gen = -7 \\ 
   & \hyperref[var:D1bb:1:1]{D1bb\_1\_1} = \grqq Angabe in Prozent"  & D1bb\_1\_1\_gen = \hyperref[var:D1bb:1:1]{D1bb\_1\_1} \\ 
   \midrule
\textbf{D1bb\_1\_2\_gen}\phantomsection\label{D1bb:1:2:gen} & \hyperref[var:D1b:2]{D1b\_2} = "nein" & \hyperref[var:D1bb:1:2:gen]{D1bb\_1\_2\_gen} = 0 \\ 
   & \hyperref[var:D1b:1:2]{D1b\_1\_2} $  <  $ 999996 \& \hyperref[var:D1]{D1} $ < $ 999997 & D1bb\_1\_2\_gen = $ \frac{\hyperref[var:D1b:1:2]{D1b\_1\_2}}{\hyperref[var:D1]{D1}} $ \\ 
   & (\hyperref[var:D1b:1:2]{D1b\_1\_2} = " weiß nicht" \xspace ODER \hyperref[var:D1]{D1} = "weiß nicht") \& \hyperref[var:D1bb:1:2]{D1bb\_1\_2} $ \neq $ "verweigert" \& \hyperref[var:D1bb:1:2]{D1bb\_1\_2} $ \neq $ \grqq Angabe in Prozent" \& \hyperref[var:D1]{D1} $ \neq $ "verweigert"  & D1bb\_1\_2\_gen = -8 \\ 
   & (\hyperref[var:D1bb:1:2]{D1bb\_1\_2} = "verweigert" \xspace ODER \hyperref[var:D1]{D1} = "verweigert") \& \hyperref[var:D1bb:1:2]{D1bb\_1\_2} $ \neq $ \grqq Angabe in Prozent"  & D1bb\_1\_2\_gen = -7 \\ 
   & \hyperref[var:D1bb:1:2]{D1bb\_1\_2} = \grqq Angabe in Prozent"  & D1bb\_1\_2\_gen = \hyperref[var:D1bb:1:2]{D1bb\_1\_2} \\ 
   \midrule
\textbf{D1bb\_1\_3\_gen}\phantomsection\label{D1bb:1:3:gen} & \hyperref[var:D1b:3]{D1b\_3} = "nein" & \hyperref[var:D1bb:1:3:gen]{D1bb\_1\_3\_gen} = 0 \\ 
   & \hyperref[var:D1b:1:3]{D1b\_1\_3} $  <  $ 999996 \& \hyperref[var:D1]{D1} $ < $ 999997 & D1bb\_1\_3\_gen = $ \frac{\hyperref[var:D1b:1:3]{D1b\_1\_3}}{\hyperref[var:D1]{D1}} $ \\ 
   & (\hyperref[var:D1b:1:3]{D1b\_1\_3} = " weiß nicht" \xspace ODER \hyperref[var:D1]{D1} = "weiß nicht") \& \hyperref[var:D1bb:1:3]{D1bb\_1\_3} $ \neq $ "verweigert" \& \hyperref[var:D1bb:1:3]{D1bb\_1\_3} $ \neq $ \grqq Angabe in Prozent" \& \hyperref[var:D1]{D1} $ \neq $ "verweigert"  & D1bb\_1\_3\_gen = -8 \\ 
   & (\hyperref[var:D1bb:1:3]{D1bb\_1\_3} = "verweigert" \xspace ODER \hyperref[var:D1]{D1} = "verweigert") \& \hyperref[var:D1bb:1:3]{D1bb\_1\_3} $ \neq $ \grqq Angabe in Prozent"  & D1bb\_1\_3\_gen = -7 \\ 
   & \hyperref[var:D1bb:1:3]{D1bb\_1\_3} = \grqq Angabe in Prozent"  & D1bb\_1\_3\_gen = \hyperref[var:D1bb:1:3]{D1bb\_1\_3} \\ 
   \midrule
\textbf{D1bb\_1\_4\_gen}\phantomsection\label{D1bb:1:4:gen} & \hyperref[var:D1b:4]{D1b\_4} = "nein" & \hyperref[var:D1bb:1:4:gen]{D1bb\_1\_4\_gen} = 0 \\ 
   & \hyperref[var:D1b:1:4]{D1b\_1\_4} $  <  $ 999996 \& \hyperref[var:D1]{D1} $ < $ 999997 & D1bb\_1\_4\_gen = $ \frac{\hyperref[var:D1b:1:4]{D1b\_1\_4}}{\hyperref[var:D1]{D1}} $ \\ 
   & (\hyperref[var:D1b:1:4]{D1b\_1\_4} = " weiß nicht" \xspace ODER \hyperref[var:D1]{D1} = "weiß nicht") \& \hyperref[var:D1bb:1:4]{D1bb\_1\_4} $ \neq $ "verweigert" \& \hyperref[var:D1bb:1:4]{D1bb\_1\_4} $ \neq $ \grqq Angabe in Prozent" \& \hyperref[var:D1]{D1} $ \neq $ "verweigert"  & D1bb\_1\_4\_gen = -8 \\ 
   & (\hyperref[var:D1bb:1:4]{D1bb\_1\_4} = "verweigert" \xspace ODER \hyperref[var:D1]{D1} = "verweigert") \& \hyperref[var:D1bb:1:4]{D1bb\_1\_4} $ \neq $ \grqq Angabe in Prozent"  & D1bb\_1\_4\_gen = -7 \\ 
   & \hyperref[var:D1bb:1:4]{D1bb\_1\_4} = \grqq Angabe in Prozent"  & D1bb\_1\_4\_gen = \hyperref[var:D1bb:1:4]{D1bb\_1\_4} \\ 
   \midrule
\textbf{D1bb\_1\_5\_gen}\phantomsection\label{D1bb:1:5:gen} & \hyperref[var:D1b:5]{D1b\_5} = "nein" & \hyperref[var:D1bb:1:5:gen]{D1bb\_1\_5\_gen} = 0 \\ 
   & \hyperref[var:D1b:1:5]{D1b\_1\_5} $  <  $ 999996 \& \hyperref[var:D1]{D1} $ < $ 999997 & D1bb\_1\_5\_gen = $ \frac{\hyperref[var:D1b:1:5]{D1b\_1\_5}}{\hyperref[var:D1]{D1}} $ \\ 
   & (\hyperref[var:D1b:1:5]{D1b\_1\_5} = " weiß nicht" \xspace ODER \hyperref[var:D1]{D1} = "weiß nicht") \& \hyperref[var:D1bb:1:5]{D1bb\_1\_5} $ \neq $ "verweigert" \& \hyperref[var:D1bb:1:5]{D1bb\_1\_5} $ \neq $ \grqq Angabe in Prozent" \& \hyperref[var:D1]{D1} $ \neq $ "verweigert"  & D1bb\_1\_5\_gen = -8 \\ 
   & (\hyperref[var:D1bb:1:5]{D1bb\_1\_5} = "verweigert" \xspace ODER \hyperref[var:D1]{D1} = "verweigert") \& \hyperref[var:D1bb:1:5]{D1bb\_1\_5} $ \neq $ \grqq Angabe in Prozent"  & D1bb\_1\_5\_gen = -7 \\ 
   & \hyperref[var:D1bb:1:5]{D1bb\_1\_5} = \grqq Angabe in Prozent"  & D1bb\_1\_5\_gen = \hyperref[var:D1bb:1:5]{D1bb\_1\_5} \\ 
   \midrule
\textbf{D1bb\_1\_7\_gen}\phantomsection\label{D1bb:1:7:gen} & \hyperref[var:D1b:7]{D1b\_7} = "nein" & \hyperref[var:D1bb:1:7:gen]{D1bb\_1\_7\_gen} = 0 \\ 
   & \hyperref[var:D1b:1:7]{D1b\_1\_7} $  <  $ 999996 \& \hyperref[var:D1]{D1} $ < $ 999997 & D1bb\_1\_7\_gen = $ \frac{\hyperref[var:D1b:1:7]{D1b\_1\_7}}{\hyperref[var:D1]{D1}} $ \\ 
   & (\hyperref[var:D1b:1:7]{D1b\_1\_7} = " weiß nicht" \xspace ODER \hyperref[var:D1]{D1} = "weiß nicht") \& \hyperref[var:D1bb:1:7]{D1bb\_1\_7} $ \neq $ "verweigert" \& \hyperref[var:D1bb:1:7]{D1bb\_1\_7} $ \neq $ \grqq Angabe in Prozent" \& \hyperref[var:D1]{D1} $ \neq $ "verweigert"  & D1bb\_1\_7\_gen = -8 \\ 
   & (\hyperref[var:D1bb:1:7]{D1bb\_1\_7} = "verweigert" \xspace ODER \hyperref[var:D1]{D1} = "verweigert") \& \hyperref[var:D1bb:1:7]{D1bb\_1\_7} $ \neq $ \grqq Angabe in Prozent"  & D1bb\_1\_7\_gen = -7 \\ 
   & \hyperref[var:D1bb:1:7]{D1bb\_1\_7} = \grqq Angabe in Prozent"  & D1bb\_1\_7\_gen = \hyperref[var:D1bb:1:7]{D1bb\_1\_7} \\ 
   \midrule
\textbf{D1cc\_1\_1\_gen}\phantomsection\label{D1cc:1:1:gen} & \hyperref[var:D1c:1]{D1c\_1} = "nein" & \hyperref[var:D1cc:1:1:gen]{D1cc\_1\_1\_gen} = 0 \\ 
   & \hyperref[var:D1c:1:1]{D1c\_1\_1} $  <  $ 999996 \& \hyperref[var:D1]{D1} $ < $ 999997 & D1cc\_1\_1\_gen = $ \frac{\hyperref[var:D1c:1:1]{D1c\_1\_1}}{\hyperref[var:D1]{D1}} $ \\ 
   & (\hyperref[var:D1c:1:1]{D1c\_1\_1} = " weiß nicht" \xspace ODER \hyperref[var:D1]{D1} = "weiß nicht") \& \hyperref[var:D1cc:1:1]{D1cc\_1\_1} $ \neq $ "verweigert" \& \hyperref[var:D1cc:1:1]{D1cc\_1\_1} $ \neq $ \grqq Angabe in Prozent" \& \hyperref[var:D1]{D1} $ \neq $ "verweigert"  & D1cc\_1\_1\_gen = -8 \\ 
   & (\hyperref[var:D1cc:1:1]{D1cc\_1\_1} = "verweigert" \xspace ODER \hyperref[var:D1]{D1} = "verweigert") \& \hyperref[var:D1cc:1:1]{D1cc\_1\_1} $ \neq $ \grqq Angabe in Prozent"  & D1cc\_1\_1\_gen = -7 \\ 
   & \hyperref[var:D1cc:1:1]{D1cc\_1\_1} = \grqq Angabe in Prozent"  & D1cc\_1\_1\_gen = \hyperref[var:D1cc:1:1]{D1cc\_1\_1} \\ 
   \midrule
\textbf{D1cc\_1\_2\_gen}\phantomsection\label{D1cc:1:2:gen} & \hyperref[var:D1c:2]{D1c\_2} = "nein" & \hyperref[var:D1cc:1:2:gen]{D1cc\_1\_2\_gen} = 0 \\ 
   & \hyperref[var:D1c:1:2]{D1c\_1\_2} $  <  $ 999996 \& \hyperref[var:D1]{D1} $ < $ 999997 & D1cc\_1\_2\_gen = $ \frac{\hyperref[var:D1c:1:2]{D1c\_1\_2}}{\hyperref[var:D1]{D1}} $ \\ 
   & (\hyperref[var:D1c:1:2]{D1c\_1\_2} = " weiß nicht" \xspace ODER \hyperref[var:D1]{D1} = "weiß nicht") \& \hyperref[var:D1cc:1:2]{D1cc\_1\_2} $ \neq $ "verweigert" \& \hyperref[var:D1cc:1:2]{D1cc\_1\_2} $ \neq $ \grqq Angabe in Prozent" \& \hyperref[var:D1]{D1} $ \neq $ "verweigert"  & D1cc\_1\_2\_gen = -8 \\ 
   & (\hyperref[var:D1cc:1:2]{D1cc\_1\_2} = "verweigert" \xspace ODER \hyperref[var:D1]{D1} = "verweigert") \& \hyperref[var:D1cc:1:2]{D1cc\_1\_2} $ \neq $ \grqq Angabe in Prozent"  & D1cc\_1\_2\_gen = -7 \\ 
   & \hyperref[var:D1cc:1:2]{D1cc\_1\_2} = \grqq Angabe in Prozent"  & D1cc\_1\_2\_gen = \hyperref[var:D1cc:1:2]{D1cc\_1\_2} \\ 
   \midrule
\textbf{D1cc\_1\_3\_gen}\phantomsection\label{D1cc:1:3:gen} & \hyperref[var:D1c:3]{D1c\_3} = "nein" & \hyperref[var:D1cc:1:3:gen]{D1cc\_1\_3\_gen} = 0 \\ 
   & \hyperref[var:D1c:1:3]{D1c\_1\_3} $  <  $ 999996 \& \hyperref[var:D1]{D1} $ < $ 999997 & D1cc\_1\_3\_gen = $ \frac{\hyperref[var:D1c:1:3]{D1c\_1\_3}}{\hyperref[var:D1]{D1}} $ \\ 
   & (\hyperref[var:D1c:1:3]{D1c\_1\_3} = " weiß nicht" \xspace ODER \hyperref[var:D1]{D1} = "weiß nicht") \& \hyperref[var:D1cc:1:3]{D1cc\_1\_3} $ \neq $ "verweigert" \& \hyperref[var:D1cc:1:3]{D1cc\_1\_3} $ \neq $ \grqq Angabe in Prozent" \& \hyperref[var:D1]{D1} $ \neq $ "verweigert"  & D1cc\_1\_3\_gen = -8 \\ 
   & (\hyperref[var:D1cc:1:3]{D1cc\_1\_3} = "verweigert" \xspace ODER \hyperref[var:D1]{D1} = "verweigert") \& \hyperref[var:D1cc:1:3]{D1cc\_1\_3} $ \neq $ \grqq Angabe in Prozent"  & D1cc\_1\_3\_gen = -7 \\ 
   & \hyperref[var:D1cc:1:3]{D1cc\_1\_3} = \grqq Angabe in Prozent"  & D1cc\_1\_3\_gen = \hyperref[var:D1cc:1:3]{D1cc\_1\_3} \\ 
   \midrule
\textbf{D1dd\_1\_1\_gen}\phantomsection\label{D1dd:1:1:gen} & \hyperref[var:D1d:1]{D1d\_1} = "nein" & \hyperref[var:D1dd:1:1:gen]{D1dd\_1\_1\_gen} = 0 \\ 
   & \hyperref[var:D1d:1:1]{D1d\_1\_1} $  <  $ 999996 \& \hyperref[var:D1]{D1} $ < $ 999997 & D1dd\_1\_1\_gen = $ \frac{\hyperref[var:D1d:1:1]{D1d\_1\_1}}{\hyperref[var:D1]{D1}} $ \\ 
   & (\hyperref[var:D1d:1:1]{D1d\_1\_1} = " weiß nicht" \xspace ODER \hyperref[var:D1]{D1} = "weiß nicht") \& \hyperref[var:D1dd:1:1]{D1dd\_1\_1} $ \neq $ "verweigert" \& \hyperref[var:D1dd:1:1]{D1dd\_1\_1} $ \neq $ \grqq Angabe in Prozent" \& \hyperref[var:D1]{D1} $ \neq $ "verweigert"  & D1dd\_1\_1\_gen = -8 \\ 
   & (\hyperref[var:D1dd:1:1]{D1dd\_1\_1} = "verweigert" \xspace ODER \hyperref[var:D1]{D1} = "verweigert") \& \hyperref[var:D1dd:1:1]{D1dd\_1\_1} $ \neq $ \grqq Angabe in Prozent"  & D1dd\_1\_1\_gen = -7 \\ 
   & \hyperref[var:D1dd:1:1]{D1dd\_1\_1} = \grqq Angabe in Prozent"  & D1dd\_1\_1\_gen = \hyperref[var:D1dd:1:1]{D1dd\_1\_1} \\ 
   \midrule
\textbf{D1dd\_1\_2\_gen}\phantomsection\label{D1dd:1:2:gen} & \hyperref[var:D1d:2]{D1d\_2} = "nein" & \hyperref[var:D1dd:1:2:gen]{D1dd\_1\_2\_gen} = 0 \\ 
   & \hyperref[var:D1d:1:2]{D1d\_1\_2} $  <  $ 999996 \& \hyperref[var:D1]{D1} $ < $ 999997 & D1dd\_1\_2\_gen = $ \frac{\hyperref[var:D1d:1:2]{D1d\_1\_2}}{\hyperref[var:D1]{D1}} $ \\ 
   & (\hyperref[var:D1d:1:2]{D1d\_1\_2} = " weiß nicht" \xspace ODER \hyperref[var:D1]{D1} = "weiß nicht") \& \hyperref[var:D1dd:1:2]{D1dd\_1\_2} $ \neq $ "verweigert" \& \hyperref[var:D1dd:1:2]{D1dd\_1\_2} $ \neq $ \grqq Angabe in Prozent" \& \hyperref[var:D1]{D1} $ \neq $ "verweigert"  & D1dd\_1\_2\_gen = -8 \\ 
   & (\hyperref[var:D1dd:1:2]{D1dd\_1\_2} = "verweigert" \xspace ODER \hyperref[var:D1]{D1} = "verweigert") \& \hyperref[var:D1dd:1:2]{D1dd\_1\_2} $ \neq $ \grqq Angabe in Prozent"  & D1dd\_1\_2\_gen = -7 \\ 
   & \hyperref[var:D1dd:1:2]{D1dd\_1\_2} = \grqq Angabe in Prozent"  & D1dd\_1\_2\_gen = \hyperref[var:D1dd:1:2]{D1dd\_1\_2} \\ 
   \midrule
\textbf{D1dd\_1\_3\_gen}\phantomsection\label{D1dd:1:3:gen} & \hyperref[var:D1d:3]{D1d\_3} = "nein" & \hyperref[var:D1dd:1:3:gen]{D1dd\_1\_3\_gen} = 0 \\ 
   & \hyperref[var:D1d:1:3]{D1d\_1\_3} $  <  $ 999996 \& \hyperref[var:D1]{D1} $ < $ 999997 & D1dd\_1\_3\_gen = $ \frac{\hyperref[var:D1d:1:3]{D1d\_1\_3}}{\hyperref[var:D1]{D1}} $ \\ 
   & (\hyperref[var:D1d:1:3]{D1d\_1\_3} = " weiß nicht" \xspace ODER \hyperref[var:D1]{D1} = "weiß nicht") \& \hyperref[var:D1dd:1:3]{D1dd\_1\_3} $ \neq $ "verweigert" \& \hyperref[var:D1dd:1:3]{D1dd\_1\_3} $ \neq $ \grqq Angabe in Prozent" \& \hyperref[var:D1]{D1} $ \neq $ "verweigert"  & D1dd\_1\_3\_gen = -8 \\ 
   & (\hyperref[var:D1dd:1:3]{D1dd\_1\_3} = "verweigert" \xspace ODER \hyperref[var:D1]{D1} = "verweigert") \& \hyperref[var:D1dd:1:3]{D1dd\_1\_3} $ \neq $ \grqq Angabe in Prozent"  & D1dd\_1\_3\_gen = -7 \\ 
   & \hyperref[var:D1dd:1:3]{D1dd\_1\_3} = \grqq Angabe in Prozent"  & D1dd\_1\_3\_gen = \hyperref[var:D1dd:1:3]{D1dd\_1\_3} \\ 
   \midrule
\textbf{D1ee\_1\_gen}\phantomsection\label{D1ee:1:gen} & \hyperref[var:D1e]{D1e} $  <  $ 999996 \& \hyperref[var:D1]{D1} $ < $ 999997 & D1ee\_1\_gen = $ \frac{\hyperref[var:D1e]{D1e}}{\hyperref[var:D1]{D1}} $ \\ 
   & (\hyperref[var:D1e]{D1e} = " weiß nicht" \xspace ODER \hyperref[var:D1]{D1} = "weiß nicht") \& \hyperref[var:D1ee:1]{D1ee\_1} $ \neq $ "verweigert" \& \hyperref[var:D1ee:1]{D1ee\_1} $ \neq $ \grqq Angabe in Prozent" \& \hyperref[var:D1]{D1} $ \neq $ "verweigert"  & D1ee\_1\_gen = -8 \\ 
   & (\hyperref[var:D1ee:1]{D1ee\_1} = "verweigert" \xspace ODER \hyperref[var:D1]{D1} = "verweigert") \& \hyperref[var:D1ee:1]{D1ee\_1} $ \neq $ \grqq Angabe in Prozent"  & D1ee\_1\_gen = -7 \\ 
   & \hyperref[var:D1ee:1]{D1ee\_1} = \grqq Angabe in Prozent"  & D1ee\_1\_gen = \hyperref[var:D1ee:1]{D1ee\_1} \\ 
   \midrule
\textbf{D1ff\_1\_gen}\phantomsection\label{D1ff:1:gen} & \hyperref[var:D1f]{D1f} $  <  $ 999996 \& \hyperref[var:D1]{D1} $ < $ 999997 & D1ff\_1\_gen = $ \frac{\hyperref[var:D1f]{D1f}}{\hyperref[var:D1]{D1}} $ \\ 
   & (\hyperref[var:D1f]{D1f} = " weiß nicht" \xspace ODER \hyperref[var:D1]{D1} = "weiß nicht") \& \hyperref[var:D1ff:1]{D1ff\_1} $ \neq $ "verweigert" \& \hyperref[var:D1ff:1]{D1ff\_1} $ \neq $ \grqq Angabe in Prozent" \& \hyperref[var:D1]{D1} $ \neq $ "verweigert"  & D1ff\_1\_gen = -8 \\ 
   & (\hyperref[var:D1ff:1]{D1ff\_1} = "verweigert" \xspace ODER \hyperref[var:D1]{D1} = "verweigert") \& \hyperref[var:D1ff:1]{D1ff\_1} $ \neq $ \grqq Angabe in Prozent"  & D1ff\_1\_gen = -7 \\ 
   & \hyperref[var:D1ff:1]{D1ff\_1} = \grqq Angabe in Prozent"  & D1ff\_1\_gen = \hyperref[var:D1ff:1]{D1ff\_1} \\ 
   \midrule
\textbf{D1gg\_1\_gen}\phantomsection\label{D1gg:1:gen} & \hyperref[var:D1g]{D1g} $  <  $ 999996 \& \hyperref[var:D1]{D1} $ < $ 999997 & D1gg\_1\_gen = $ \frac{\hyperref[var:D1g]{D1g}}{\hyperref[var:D1]{D1}} $ \\ 
   & (\hyperref[var:D1g]{D1g} = " weiß nicht" \xspace ODER \hyperref[var:D1]{D1} = "weiß nicht") \& \hyperref[var:D1gg:1]{D1gg\_1} $ \neq $ "verweigert" \& \hyperref[var:D1gg:1]{D1gg\_1} $ \neq $ \grqq Angabe in Prozent" \& \hyperref[var:D1]{D1} $ \neq $ "verweigert"  & D1gg\_1\_gen = -8 \\ 
   & (\hyperref[var:D1gg:1]{D1gg\_1} = "verweigert" \xspace ODER \hyperref[var:D1]{D1} = "verweigert") \& \hyperref[var:D1gg:1]{D1gg\_1} $ \neq $ \grqq Angabe in Prozent"  & D1gg\_1\_gen = -7 \\ 
   & \hyperref[var:D1gg:1]{D1gg\_1} = \grqq Angabe in Prozent"  & D1gg\_1\_gen = \hyperref[var:D1gg:1]{D1gg\_1} \\ 
   \midrule
\textbf{F3\_1\_gen}\phantomsection\label{F3:1:gen} & \hyperref[var:F3]{F3} $  <  $ 999996 \& \hyperref[var:D1]{D1} $ < $ 999997 & F3\_1\_gen = $ \frac{\hyperref[var:F3]{F3}}{\hyperref[var:D1]{D1}} $ \\ 
   & (\hyperref[var:F3]{F3} = " weiß nicht" \xspace ODER \hyperref[var:D1]{D1} = "weiß nicht") \& \hyperref[var:F3:1]{F3\_1} $ \neq $ "verweigert" \& \hyperref[var:F3:1]{F3\_1} $ \neq $ \grqq Angabe in Prozent" \& \hyperref[var:D1]{D1} $ \neq $ "verweigert"  & F3\_1\_gen = -8 \\ 
   & (\hyperref[var:F3:1]{F3\_1} = "verweigert" \xspace ODER \hyperref[var:D1]{D1} = "verweigert") \& \hyperref[var:F3:1]{F3\_1} $ \neq $ \grqq Angabe in Prozent"  & F3\_1\_gen = -7 \\ 
   & \hyperref[var:F3:1]{F3\_1} = \grqq Angabe in Prozent"  & F3\_1\_gen = \hyperref[var:F3:1]{F3\_1} \\ 
   \midrule
\textbf{M4\_gen}\phantomsection\label{M4:gen} & \hyperref[var:M4]{M4} $ \neq $ " weiß nicht" \xspace \& \hyperref[var:M4]{M4} $ \neq $ " verweigert" \xspace \& \hyperref[var:M3]{M3} $ \neq $ "verweigert" \& \hyperref[var:M3]{M3} $ \neq $ "weiß nicht" \&  & M4\_gen = $ \frac{\hyperref[var:M4]{M4}}{\hyperref[var:M3]{M3} } $ \\ 
   & (\hyperref[var:M3]{M3} = "verweigert" \xspace ODER \hyperref[var:M3]{M3} = "weiß nicht" ) \& \hyperref[var:M4]{M4} $ \neq $ \textit{missing by design} & M4\_gen = \hyperref[var:M3]{M3} \\ 
   \midrule
\textbf{M5a\_gen}\phantomsection\label{M5a:gen} & \hyperref[var:M5a]{M5a} $ \neq $ " weiß nicht" \xspace \& \hyperref[var:M5a]{M5a} $ \neq $ " verweigert" \xspace \& \hyperref[var:M3]{M3} $ \neq $ "verweigert" \& \hyperref[var:M3]{M3} $ \neq $ "weiß nicht" \&  & M5a\_gen = $ \frac{\hyperref[var:M5a]{M5a}}{\hyperref[var:M3]{M3} } $ \\ 
   & (\hyperref[var:M3]{M3} = "verweigert" \xspace ODER \hyperref[var:M3]{M3} = "weiß nicht" ) \& \hyperref[var:M5a]{M5a} $ \neq $ \textit{missing by design} & M5a\_gen = \hyperref[var:M3]{M3} \\ 
   \midrule
\textbf{M5b\_gen}\phantomsection\label{M5b:gen} & \hyperref[var:M5b]{M5b} $ \neq $ " weiß nicht" \xspace \& \hyperref[var:M5b]{M5b} $ \neq $ " verweigert" \xspace \& \hyperref[var:M3]{M3} $ \neq $ "verweigert" \& \hyperref[var:M3]{M3} $ \neq $ "weiß nicht" \&  & M5b\_gen = $ \frac{\hyperref[var:M5b]{M5b}}{\hyperref[var:M3]{M3} } $ \\ 
   & (\hyperref[var:M3]{M3} = "verweigert" \xspace ODER \hyperref[var:M3]{M3} = "weiß nicht" ) \& \hyperref[var:M5b]{M5b} $ \neq $ \textit{missing by design} & M5b\_gen = \hyperref[var:M3]{M3} \\ 
   \midrule
\textbf{M6\_gen}\phantomsection\label{M6:gen} & \hyperref[var:M6]{M6} $ \neq $ " weiß nicht" \xspace \& \hyperref[var:M6]{M6} $ \neq $ " verweigert" \xspace \& \hyperref[var:M3]{M3} $ \neq $ "verweigert" \& \hyperref[var:M3]{M3} $ \neq $ "weiß nicht" \&  & M6\_gen = $ \frac{\hyperref[var:M6]{M6}}{\hyperref[var:M3]{M3} } $ \\ 
   & (\hyperref[var:M3]{M3} = "verweigert" \xspace ODER \hyperref[var:M3]{M3} = "weiß nicht" ) \& \hyperref[var:M6]{M6} $ \neq $ \textit{missing by design} & M6\_gen = \hyperref[var:M3]{M3} \\ 
   \midrule
\textbf{M6a\_gen}\phantomsection\label{M6a:gen} & \hyperref[var:M6a]{M6a} $ \neq $ " weiß nicht" \xspace \& \hyperref[var:M6a]{M6a} $ \neq $ " verweigert" \xspace \& \hyperref[var:M3]{M3} $ \neq $ "verweigert" \& \hyperref[var:M3]{M3} $ \neq $ "weiß nicht" \&  & M6a\_gen = $ \frac{\hyperref[var:M6a]{M6a}}{\hyperref[var:M3]{M3} } $ \\ 
   & (\hyperref[var:M3]{M3} = "verweigert" \xspace ODER \hyperref[var:M3]{M3} = "weiß nicht" ) \& \hyperref[var:M6a]{M6a} $ \neq $ \textit{missing by design} & M6a\_gen = \hyperref[var:M3]{M3} \\ 
   \midrule
\textbf{M6b\_gen}\phantomsection\label{M6b:gen} & \hyperref[var:M6b]{M6b} $ \neq $ " weiß nicht" \xspace \& \hyperref[var:M6b]{M6b} $ \neq $ " verweigert" \xspace \& \hyperref[var:M3]{M3} $ \neq $ "verweigert" \& \hyperref[var:M3]{M3} $ \neq $ "weiß nicht" \&  & M6b\_gen = $ \frac{\hyperref[var:M6b]{M6b}}{\hyperref[var:M3]{M3} } $ \\ 
   & (\hyperref[var:M3]{M3} = "verweigert" \xspace ODER \hyperref[var:M3]{M3} = "weiß nicht" ) \& \hyperref[var:M6b]{M6b} $ \neq $ \textit{missing by design} & M6b\_gen = \hyperref[var:M3]{M3} \\ 
   \midrule
\textbf{M6c\_gen}\phantomsection\label{M6c:gen} & \hyperref[var:M6c]{M6c} $ \neq $ " weiß nicht" \xspace \& \hyperref[var:M6c]{M6c} $ \neq $ " verweigert" \xspace \& \hyperref[var:M3]{M3} $ \neq $ "verweigert" \& \hyperref[var:M3]{M3} $ \neq $ "weiß nicht" \&  & M6c\_gen = $ \frac{\hyperref[var:M6c]{M6c}}{\hyperref[var:M3]{M3} } $ \\ 
   & (\hyperref[var:M3]{M3} = "verweigert" \xspace ODER \hyperref[var:M3]{M3} = "weiß nicht" ) \& \hyperref[var:M6c]{M6c} $ \neq $ \textit{missing by design} & M6c\_gen = \hyperref[var:M3]{M3} \\ 
   \midrule
\textbf{M6e\_gen}\phantomsection\label{M6e:gen} & \hyperref[var:M6e]{M6e} $ \neq $ " weiß nicht" \xspace \& \hyperref[var:M6e]{M6e} $ \neq $ " verweigert" \xspace \& \hyperref[var:M3]{M3} $ \neq $ "verweigert" \& \hyperref[var:M3]{M3} $ \neq $ "weiß nicht" \&  & M6e\_gen = $ \frac{\hyperref[var:M6e]{M6e}}{\hyperref[var:M3]{M3} } $ \\ 
   & (\hyperref[var:M3]{M3} = "verweigert" \xspace ODER \hyperref[var:M3]{M3} = "weiß nicht" ) \& \hyperref[var:M6e]{M6e} $ \neq $ \textit{missing by design} & M6e\_gen = \hyperref[var:M3]{M3} \\ 
   \midrule
\textbf{M6f\_gen}\phantomsection\label{M6f:gen} & \hyperref[var:M6f]{M6f} $ \neq $ " weiß nicht" \xspace \& \hyperref[var:M6f]{M6f} $ \neq $ " verweigert" \xspace \& \hyperref[var:M3]{M3} $ \neq $ "verweigert" \& \hyperref[var:M3]{M3} $ \neq $ "weiß nicht" \&  & M6f\_gen = $ \frac{\hyperref[var:M6f]{M6f}}{\hyperref[var:M3]{M3} } $ \\ 
   & (\hyperref[var:M3]{M3} = "verweigert" \xspace ODER \hyperref[var:M3]{M3} = "weiß nicht" ) \& \hyperref[var:M6f]{M6f} $ \neq $ \textit{missing by design} & M6f\_gen = \hyperref[var:M3]{M3} \\ 
   \midrule
\textbf{M6g\_gen}\phantomsection\label{M6g:gen} & \hyperref[var:M6g]{M6g} $ \neq $ " weiß nicht" \xspace \& \hyperref[var:M6g]{M6g} $ \neq $ " verweigert" \xspace \& \hyperref[var:M3]{M3} $ \neq $ "verweigert" \& \hyperref[var:M3]{M3} $ \neq $ "weiß nicht" \&  & M6g\_gen = $ \frac{\hyperref[var:M6g]{M6g}}{\hyperref[var:M3]{M3} } $ \\ 
   & (\hyperref[var:M3]{M3} = "verweigert" \xspace ODER \hyperref[var:M3]{M3} = "weiß nicht" ) \& \hyperref[var:M6g]{M6g} $ \neq $ \textit{missing by design} & M6g\_gen = \hyperref[var:M3]{M3} \\ 
   \midrule
\textbf{M7\_gen}\phantomsection\label{M7:gen} & \hyperref[var:M7]{M7} $ \neq $ " weiß nicht" \xspace \& \hyperref[var:M7]{M7} $ \neq $ " verweigert" \xspace \& \hyperref[var:M3]{M3} $ \neq $ "verweigert" \& \hyperref[var:M3]{M3} $ \neq $ "weiß nicht" \&  & M7\_gen = $ \frac{\hyperref[var:M7]{M7}}{\hyperref[var:M3]{M3} } $ \\ 
   & (\hyperref[var:M3]{M3} = "verweigert" \xspace ODER \hyperref[var:M3]{M3} = "weiß nicht" ) \& \hyperref[var:M7]{M7} $ \neq $ \textit{missing by design} & M7\_gen = \hyperref[var:M3]{M3} \\ 
   \midrule
\textbf{M8\_gen}\phantomsection\label{M8:gen} & \hyperref[var:M8]{M8} $ \neq $ " weiß nicht" \xspace \& \hyperref[var:M8]{M8} $ \neq $ " verweigert" \xspace \& \hyperref[var:M3]{M3} $ \neq $ "verweigert" \& \hyperref[var:M3]{M3} $ \neq $ "weiß nicht" \&  & M8\_gen = $ \frac{\hyperref[var:M8]{M8}}{\hyperref[var:M3]{M3} } $ \\ 
   & (\hyperref[var:M3]{M3} = "verweigert" \xspace ODER \hyperref[var:M3]{M3} = "weiß nicht" ) \& \hyperref[var:M8]{M8} $ \neq $ \textit{missing by design} & M8\_gen = \hyperref[var:M3]{M3} \\ 
   \midrule
\textbf{M11\_gen}\phantomsection\label{M11:gen} & \hyperref[var:M11]{M11} $ \neq $ " weiß nicht" \xspace \& \hyperref[var:M11]{M11} $ \neq $ " verweigert" \xspace \& \hyperref[var:M3]{M3} $ \neq $ "verweigert" \& \hyperref[var:M3]{M3} $ \neq $ "weiß nicht" \&  & M11\_gen = $ \frac{\hyperref[var:M11]{M11}}{\hyperref[var:M3]{M3} } $ \\ 
   & (\hyperref[var:M3]{M3} = "verweigert" \xspace ODER \hyperref[var:M3]{M3} = "weiß nicht" ) \& \hyperref[var:M11]{M11} $ \neq $ \textit{missing by design} & M11\_gen = \hyperref[var:M3]{M3} \\ 
   \midrule
\textbf{M11a\_gen}\phantomsection\label{M11a:gen} & \hyperref[var:M11a]{M11a} $ \neq $ " weiß nicht" \xspace \& \hyperref[var:M11a]{M11a} $ \neq $ " verweigert" \xspace \& \hyperref[var:M3]{M3} $ \neq $ "verweigert" \& \hyperref[var:M3]{M3} $ \neq $ "weiß nicht" \&  & M11a\_gen = $ \frac{\hyperref[var:M11a]{M11a}}{\hyperref[var:M3]{M3} } $ \\ 
   & (\hyperref[var:M3]{M3} = "verweigert" \xspace ODER \hyperref[var:M3]{M3} = "weiß nicht" ) \& \hyperref[var:M11a]{M11a} $ \neq $ \textit{missing by design} & M11a\_gen = \hyperref[var:M3]{M3} \\ 
   \midrule
\textbf{M11b\_gen}\phantomsection\label{M11b:gen} & \hyperref[var:M11b]{M11b} $ \neq $ " weiß nicht" \xspace \& \hyperref[var:M11b]{M11b} $ \neq $ " verweigert" \xspace \& \hyperref[var:M3]{M3} $ \neq $ "verweigert" \& \hyperref[var:M3]{M3} $ \neq $ "weiß nicht" \&  & M11b\_gen = $ \frac{\hyperref[var:M11b]{M11b}}{\hyperref[var:M3]{M3} } $ \\ 
   & (\hyperref[var:M3]{M3} = "verweigert" \xspace ODER \hyperref[var:M3]{M3} = "weiß nicht" ) \& \hyperref[var:M11b]{M11b} $ \neq $ \textit{missing by design} & M11b\_gen = \hyperref[var:M3]{M3} \\ 
  
	
\end{longtable}

\subsection{Vergröberte Variablen}\label{var_kategorisiert}

Einige Variablen des Rohdatensatzes enthalten einzelne Angaben, die die Anonymit?t des Betriebes bzw. des Betriebsrates gefährden. Aus diesen Variablen wurden klassierte Daten generiert. Diese Variablen sind im Datensatz des Scientific Use Files durch das Suffix \glqq \_kat \grqq\xspace gekennzeichnet, auch ihr Variablenlabel enthält einen entsprechenden Hinweis (\grqq vergröbert\grqq ). Dies betrifft im Einzelnen:

\begin{enumerate}

\item \hyperref[var:D1]{D1}: Anzahl der Beschäftigten im Betrieb 
\item \hyperref[var:I7]{I7}: Höhe des Krankenstandes am 01.12.2014 in Prozent
\item \hyperref[var:M1]{M1}: Existenzdauer des Betriebsrats in Jahren
\item \hyperref[var:M2]{M2}: Wahljahr des Betriebsrats
\item \hyperref[var:M3]{M3}: Anzahl der Betriebsratsmitglieder
\item \hyperref[var:M13]{M13}: Alter des Betriebsratsvorsitzenden
\item \hyperref[var:R1a]{R1a}: Anzahl der Besch?ftigten im Gesamtunternehmen

\end{enumerate}

\subsection{Gelöschte Variablen}\label{var_geloescht}

Dem Scientific Use File wurden gegenüber dem Rohdatensatz eine Reihe von Variablen entnommen. Damit wird zunächst die Anonymität der Zielpersonen gewährleistet, deren Angaben vor allem zur methodischen Qualit?tssicherung der Fragen erhoben werden. Da es sich jedoch um eine Gremienbefragung handelt, für die Zielperson lediglich als Repr?sentant fungiert, sind diese Angaben für die inhaltliche Forschung nicht notwendig. Auch die Entfernung offener Angaben dient, da sich darunter exakte Wortlaute finden, der Anonymisierung des Datensatzes. 

Weiter wurden Variablen zur Übersichtlichkeit des Datensatzes entnommen. Darunter fallen die Methodenvariablen (z.B. die Reihenfolge, in der gewisse Fragen eingespielt wurden), aber auch alle Variablen, die zu \hyperref[var_generiert]{generierten} oder \hyperref[var_kategorisiert]{kategorisierten} Variablen weiterverarbeitet wurden.

\begin{longtable}{!{\color{black}\vline width 1pt} L{2.5cm} !{\color{black}\vline width 1pt} L{2.5cm} | L{11.75cm} !{\color{black}\vline width 1pt}  }
	
	\toprule
	\textbf{Löschgrund} & \textbf{Variable} & \textbf{Variablenbezeichnung}  \\ 
	\midrule
	\endfirsthead

	\toprule
		\textbf{Löschgrund} & \textbf{Variable} & \textbf{Variablenbezeichnung}  \\ 
	\midrule
	\endhead
	
	\midrule
	
	\endfoot
	\bottomrule
	\endlastfoot
		
	% latex table generated in R 3.6.3 by xtable 1.8-4 package
% Tue Mar 17 12:15:16 2020
 \multirow{2}{2.5cm}{\textbf{Zielpersonen-angaben}} & \hyperref[var:C0a]{C0a} & \textbf{Zugehörigkeit der Zielperson zum Gesamtbetriebsrat} \\ 
   & \hyperref[var:C0b]{C0b} & \textbf{Zugehörigkeit der Zielperson zum Konzernbetriebsrat} \\ 
   & \hyperref[var:C0c]{C0c} & \textbf{Zugehörigkeit der Zielperson zum Euro-Betriebsrat} \\ 
   & \hyperref[var:Q1]{Q1} & \textbf{Geschlecht der Zielperson} \\ 
   & \hyperref[var:Q2]{Q2} & \textbf{Position der Zielperson im Betriebsrat} \\ 
   & \hyperref[var:Q3]{Q3} & \textbf{Betriebsratsmitgliedschaft seit...} \\ 
   & \hyperref[var:Q4]{Q4} & \textbf{In der jetzigen Funktion im Betriebsrat seit...} \\ 
   & \hyperref[var:Q5]{Q5} & \textbf{Alter der Zielperson} \\ 
   & \hyperref[var:Q6]{Q6} & \textbf{Gewerkschaftsmitgliedschaft} \\ 
   & \hyperref[var:Q7]{Q7} & \textbf{Bezeichnung der Gewerkschaft} \\ 
   & \hyperref[var:S2]{S2} & \textbf{Name und Betriebsadresse des Betriebsrats} \\ 
   & \hyperref[var:S3]{S3} & \textbf{Danksagung} \\ 
   \midrule
\multirow{2}{2.5cm}{\textbf{Methoden-variablen}} & \hyperref[var:splithalf]{splithalf} & \textbf{SPLITHALF} \\ 
   & \hyperref[var:langkurz]{langkurz} & \textbf{Kennung Lang-/Kurzversion Fragebogen} \\ 
   & \hyperref[var:ORDER01:G1]{ORDER01\_G1} & \textbf{G1\_1: 01.Position Item-Aufruf} \\ 
   & \hyperref[var:ORDER02:G1]{ORDER02\_G1} & \textbf{G1\_1: 02.Position Item-Aufruf} \\ 
   & \hyperref[var:ORDER03:G1]{ORDER03\_G1} & \textbf{G1\_1: 03.Position Item-Aufruf} \\ 
   & \hyperref[var:ORDER04:G1]{ORDER04\_G1} & \textbf{G1\_1: 04.Position Item-Aufruf} \\ 
   & \hyperref[var:ORDER05:G1]{ORDER05\_G1} & \textbf{G1\_1: 05.Position Item-Aufruf} \\ 
   & \hyperref[var:ORDER06:G1]{ORDER06\_G1} & \textbf{G1\_1: 06.Position Item-Aufruf} \\ 
   & \hyperref[var:ORDER07:G1]{ORDER07\_G1} & \textbf{G1\_1: 07.Position Item-Aufruf} \\ 
   & \hyperref[var:ORDER08:G1]{ORDER08\_G1} & \textbf{G1\_1: 08.Position Item-Aufruf} \\ 
   & \hyperref[var:ORDER09:G1]{ORDER09\_G1} & \textbf{G1\_1: 09.Position Item-Aufruf} \\ 
   & \hyperref[var:ORDER10:G1]{ORDER10\_G1} & \textbf{G1\_1: 10.Position Item-Aufruf} \\ 
   & \hyperref[var:ORDER11:G1]{ORDER11\_G1} & \textbf{G1\_1: 11.Position Item-Aufruf} \\ 
   & \hyperref[var:ORDER12:G1]{ORDER12\_G1} & \textbf{G1\_1: 12.Position Item-Aufruf} \\ 
   & \hyperref[var:ORDER13:G1]{ORDER13\_G1} & \textbf{G1\_1: 13.Position Item-Aufruf} \\ 
   & \hyperref[var:ORDER14:G1]{ORDER14\_G1} & \textbf{G1\_1: 14.Position Item-Aufruf} \\ 
   & \hyperref[var:ORDER15:G1]{ORDER15\_G1} & \textbf{G1\_1: 15.Position Item-Aufruf} \\ 
   & \hyperref[var:ORDER16:G1]{ORDER16\_G1} & \textbf{G1\_1: 16.Position Item-Aufruf} \\ 
   & \hyperref[var:ORDER17:G1]{ORDER17\_G1} & \textbf{G1\_1: 17.Position Item-Aufruf} \\ 
   & \hyperref[var:ORDER18:G1]{ORDER18\_G1} & \textbf{G1\_1: 18.Position Item-Aufruf} \\ 
   & \hyperref[var:ORDER19:G1]{ORDER19\_G1} & \textbf{G1\_1: 19.Position Item-Aufruf} \\ 
   & \hyperref[var:ORDER20:G1]{ORDER20\_G1} & \textbf{G1\_1: 20.Position Item-Aufruf} \\ 
   & \hyperref[var:ORDER21:G1]{ORDER21\_G1} & \textbf{G1\_1: 21.Position Item-Aufruf} \\ 
   & \hyperref[var:ORDER22:G1]{ORDER22\_G1} & \textbf{G1\_1: 22.Position Item-Aufruf} \\ 
   & \hyperref[var:ORDER23:G1]{ORDER23\_G1} & \textbf{G1\_1: 23.Position Item-Aufruf} \\ 
   & \hyperref[var:ORDER24:G1]{ORDER24\_G1} & \textbf{G1\_1: 24.Position Item-Aufruf} \\ 
   & \hyperref[var:ORDER25:G1]{ORDER25\_G1} & \textbf{G1\_1: 25.Position Item-Aufruf} \\ 
   & \hyperref[var:ORDER26:G1]{ORDER26\_G1} & \textbf{G1\_1: 26.Position Item-Aufruf} \\ 
   & \hyperref[var:ORDER27:G1]{ORDER27\_G1} & \textbf{G1\_1: 27.Position Item-Aufruf} \\ 
   & \hyperref[var:ORDER28:G1]{ORDER28\_G1} & \textbf{G1\_1: 28.Position Item-Aufruf} \\ 
   & \hyperref[var:ORDER29:G1]{ORDER29\_G1} & \textbf{G1\_1: 29.Position Item-Aufruf} \\ 
   & \hyperref[var:ORDER30:G1]{ORDER30\_G1} & \textbf{G1\_1: 30.Position Item-Aufruf} \\ 
   & \hyperref[var:ORDER31:G1]{ORDER31\_G1} & \textbf{G1\_1: 31.Position Item-Aufruf} \\ 
   & \hyperref[var:ORDER32:G1]{ORDER32\_G1} & \textbf{G1\_1: 32.Position Item-Aufruf} \\ 
   & \hyperref[var:ORDER33:G1]{ORDER33\_G1} & \textbf{G1\_1: 33.Position Item-Aufruf} \\ 
   & \hyperref[var:ORDER34:G1]{ORDER34\_G1} & \textbf{G1\_1: 34.Position Item-Aufruf} \\ 
   & \hyperref[var:ORDER35:G1]{ORDER35\_G1} & \textbf{G1\_1: 35.Position Item-Aufruf} \\ 
   & \hyperref[var:ORDER36:G1]{ORDER36\_G1} & \textbf{G1\_1: 36.Position Item-Aufruf} \\ 
   & \hyperref[var:ORDER37:G1]{ORDER37\_G1} & \textbf{G1\_1: 37.Position Item-Aufruf} \\ 
   & \hyperref[var:Order01:D1c]{Order01\_D1c} & \textbf{D1c: 1.Position Item-Aufruf} \\ 
   & \hyperref[var:Order02:D1c]{Order02\_D1c} & \textbf{D1c: 2.Position Item-Aufruf} \\ 
   & \hyperref[var:Order03:D1c]{Order03\_D1c} & \textbf{D1c: 3.Position Item-Aufruf} \\ 
   & \hyperref[var:ORDER01:I3]{ORDER01\_I3} & \textbf{I3: 01.Position Item-Aufruf} \\ 
   & \hyperref[var:ORDER02:I3]{ORDER02\_I3} & \textbf{I3: 02.Position Item-Aufruf} \\ 
   & \hyperref[var:ORDER03:I3]{ORDER03\_I3} & \textbf{I3: 03.Position Item-Aufruf} \\ 
   & \hyperref[var:ORDER04:I3]{ORDER04\_I3} & \textbf{I3: 04.Position Item-Aufruf} \\ 
   & \hyperref[var:ORDER05:I3]{ORDER05\_I3} & \textbf{I3: 05.Position Item-Aufruf} \\ 
   & \hyperref[var:ORDER06:I3]{ORDER06\_I3} & \textbf{I3: 06.Position Item-Aufruf} \\ 
   & \hyperref[var:ORDER07:I3]{ORDER07\_I3} & \textbf{I3: 07.Position Item-Aufruf} \\ 
   & \hyperref[var:ORDER01:J7]{ORDER01\_J7} & \textbf{J7: 01.Position Item-Aufruf} \\ 
   & \hyperref[var:ORDER02:J7]{ORDER02\_J7} & \textbf{J7: 02.Position Item-Aufruf} \\ 
   & \hyperref[var:ORDER03:J7]{ORDER03\_J7} & \textbf{J7: 03.Position Item-Aufruf} \\ 
   & \hyperref[var:ORDER04:J7]{ORDER04\_J7} & \textbf{J7: 04.Position Item-Aufruf} \\ 
   & \hyperref[var:ORDER05:J7]{ORDER05\_J7} & \textbf{J7: 05.Position Item-Aufruf} \\ 
   & \hyperref[var:ORDER06:J7]{ORDER06\_J7} & \textbf{J7: 06.Position Item-Aufruf} \\ 
   & \hyperref[var:ORDER01:K6a]{ORDER01\_K6a} & \textbf{K6a: 01.Position Item-Aufruf} \\ 
   & \hyperref[var:ORDER02:K6a]{ORDER02\_K6a} & \textbf{K6a: 02.Position Item-Aufruf} \\ 
   & \hyperref[var:ORDER03:K6a]{ORDER03\_K6a} & \textbf{K6a: 03.Position Item-Aufruf} \\ 
   & \hyperref[var:ORDER04:K6a]{ORDER04\_K6a} & \textbf{K6a: 04.Position Item-Aufruf} \\ 
   & \hyperref[var:ORDER05:K6a]{ORDER05\_K6a} & \textbf{K6a: 05.Position Item-Aufruf} \\ 
   & \hyperref[var:ORDER06:K6a]{ORDER06\_K6a} & \textbf{K6a: 06.Position Item-Aufruf} \\ 
   & \hyperref[var:ORDER01:K7a]{ORDER01\_K7a} & \textbf{K7a: 01.Position Item-Aufruf} \\ 
   & \hyperref[var:ORDER02:K7a]{ORDER02\_K7a} & \textbf{K7a: 02.Position Item-Aufruf} \\ 
   & \hyperref[var:ORDER03:K7a]{ORDER03\_K7a} & \textbf{K7a: 03.Position Item-Aufruf} \\ 
   & \hyperref[var:ORDER04:K7a]{ORDER04\_K7a} & \textbf{K7a: 04.Position Item-Aufruf} \\ 
   & \hyperref[var:ORDER05:K7a]{ORDER05\_K7a} & \textbf{K7a: 05.Position Item-Aufruf} \\ 
   & \hyperref[var:ORDER06:K7a]{ORDER06\_K7a} & \textbf{K7a: 06.Position Item-Aufruf} \\ 
   & \hyperref[var:ORDER01:K11]{ORDER01\_K11} & \textbf{K11: 01.Position Item-Aufruf} \\ 
   & \hyperref[var:ORDER02:K11]{ORDER02\_K11} & \textbf{K11: 02.Position Item-Aufruf} \\ 
   & \hyperref[var:ORDER03:K11]{ORDER03\_K11} & \textbf{K11: 03.Position Item-Aufruf} \\ 
   & \hyperref[var:ORDER04:K11]{ORDER04\_K11} & \textbf{K11: 04.Position Item-Aufruf} \\ 
   & \hyperref[var:ORDER05:K11]{ORDER05\_K11} & \textbf{K11: 05.Position Item-Aufruf} \\ 
   & \hyperref[var:ORDER06:K11]{ORDER06\_K11} & \textbf{K11: 06.Position Item-Aufruf} \\ 
   & \hyperref[var:ORDER07:K11]{ORDER07\_K11} & \textbf{K11: 07.Position Item-Aufruf} \\ 
   & \hyperref[var:ORDER08:K11]{ORDER08\_K11} & \textbf{K11: 08.Position Item-Aufruf} \\ 
   & \hyperref[var:ORDER09:K11]{ORDER09\_K11} & \textbf{K11: 09.Position Item-Aufruf} \\ 
   & \hyperref[var:ORDER10:K11]{ORDER10\_K11} & \textbf{K11: 10.Position Item-Aufruf} \\ 
   & \hyperref[var:ORDER11:K11]{ORDER11\_K11} & \textbf{K11: 11.Position Item-Aufruf} \\ 
   & \hyperref[var:ORDER12:K11]{ORDER12\_K11} & \textbf{K11: 12.Position Item-Aufruf} \\ 
   & \hyperref[var:ORDER13:K11]{ORDER13\_K11} & \textbf{K11: 13.Position Item-Aufruf} \\ 
   & \hyperref[var:ORDER14:K11]{ORDER14\_K11} & \textbf{K11: 14.Position Item-Aufruf} \\ 
   & \hyperref[var:ORDER15:K11]{ORDER15\_K11} & \textbf{K11: 15.Position Item-Aufruf} \\ 
   & \hyperref[var:ORDER16:K11]{ORDER16\_K11} & \textbf{K11: 16.Position Item-Aufruf} \\ 
   & \hyperref[var:ORDER17:K11]{ORDER17\_K11} & \textbf{K11: 17.Position Item-Aufruf} \\ 
   & \hyperref[var:ORDER18:K11]{ORDER18\_K11} & \textbf{K11: 18.Position Item-Aufruf} \\ 
   & \hyperref[var:ORDER19:K11]{ORDER19\_K11} & \textbf{K11: 19.Position Item-Aufruf} \\ 
   & \hyperref[var:ORDER20:K11]{ORDER20\_K11} & \textbf{K11: 20.Position Item-Aufruf} \\ 
   & \hyperref[var:ORDER21:K11]{ORDER21\_K11} & \textbf{K11: 21.Position Item-Aufruf} \\ 
   & \hyperref[var:ORDER22:K11]{ORDER22\_K11} & \textbf{K11: 22.Position Item-Aufruf} \\ 
   & \hyperref[var:ORDER23:K11]{ORDER23\_K11} & \textbf{K11: 23.Position Item-Aufruf} \\ 
   & \hyperref[var:ORDER24:K11]{ORDER24\_K11} & \textbf{K11: 24.Position Item-Aufruf} \\ 
   & \hyperref[var:ORDER25:K11]{ORDER25\_K11} & \textbf{K11: 25.Position Item-Aufruf} \\ 
   & \hyperref[var:ORDER26:K11]{ORDER26\_K11} & \textbf{K11: 26.Position Item-Aufruf} \\ 
   & \hyperref[var:ORDER27:K11]{ORDER27\_K11} & \textbf{K11: 27.Position Item-Aufruf} \\ 
   & \hyperref[var:ORDER28:K11]{ORDER28\_K11} & \textbf{K11: 28.Position Item-Aufruf} \\ 
   & \hyperref[var:ORDER01:P2]{ORDER01\_P2} & \textbf{P2: 01.Position Item-Aufruf} \\ 
   & \hyperref[var:ORDER02:P2]{ORDER02\_P2} & \textbf{P2: 02.Position Item-Aufruf} \\ 
   & \hyperref[var:ORDER03:P2]{ORDER03\_P2} & \textbf{P2: 03.Position Item-Aufruf} \\ 
   & \hyperref[var:ORDER04:P2]{ORDER04\_P2} & \textbf{P2: 04.Position Item-Aufruf} \\ 
   & \hyperref[var:ORDER05:P2]{ORDER05\_P2} & \textbf{P2: 05.Position Item-Aufruf} \\ 
   & \hyperref[var:ORDER06:P2]{ORDER06\_P2} & \textbf{P2: 06.Position Item-Aufruf} \\ 
   & \hyperref[var:ORDER07:P2]{ORDER07\_P2} & \textbf{P2: 07.Position Item-Aufruf} \\ 
   & \hyperref[var:S1]{S1} & \textbf{Panelbereitschaft} \\ 
   & \hyperref[var:S4]{S4} & \textbf{Befragungsbereitschaft des Betriebsrats} \\ 
   & \hyperref[var:Interviewdauer]{Interviewdauer} & \textbf{Interviewdauer in Minuten} \\ 
   \midrule
\multirow{2}{2.5cm}{\textbf{offene Angaben}} & \hyperref[var:C10]{C10} & \textbf{Wirtschaftszweig des Betriebes} \\ 
   & \hyperref[var:C10:o]{C10\_o} & \textbf{Wirtschaftszweig des Betriebes (offen)} \\ 
   & \hyperref[var:P3:o:5]{P3\_o\_5} & \textbf{Andere Parteien (offen)} \\ 
   \midrule
\multirow{2}{2.5cm}{\textbf{Neue Var. generiert}} & \hyperref[var:C1]{C1} & \textbf{Anzahl der Beschäftigten im Betrieb, die Gewerkschaftsmitglieder sind} \\ 
   & \hyperref[var:C1:1]{C1\_1} & \textbf{Anteil der Beschäftigten im Betrieb, die Gewerkschaftsmitglieder sind} \\ 
   & \hyperref[var:D1a]{D1a} & \textbf{Anzahl der Frauen unter den Beschäftigten} \\ 
   & \hyperref[var:D1aa]{D1aa} & \textbf{Frauenanteil unter den Beschäftigten} \\ 
   & \hyperref[var:D1b:1]{D1b\_1} & \textbf{Vorhandensein von Vollzeitbeschäftigten im Betrieb} \\ 
   & \hyperref[var:D1b:1:1]{D1b\_1\_1} & \textbf{Anzahl von Vollzeitbeschäftigten im Betrieb} \\ 
   & \hyperref[var:D1bb:1:1]{D1bb\_1\_1} & \textbf{Anteil von Vollzeitbeschäftigten im Betrieb} \\ 
   & \hyperref[var:D1b:1:2]{D1b\_1\_2} & \textbf{Anzahl von Teilzeitbeschäftigten ohne Minijobs im Betrieb} \\ 
   & \hyperref[var:D1bb:1:2]{D1bb\_1\_2} & \textbf{Anteil von Teilzeitbeschäftigten ohne Minijobs im Betrieb} \\ 
   & \hyperref[var:D1b:3]{D1b\_3} & \textbf{Vorhandensein von Minijobs bis 450 € im Betrieb} \\ 
   & \hyperref[var:D1b:1:3]{D1b\_1\_3} & \textbf{Anzahl von Minijobs bis 450 € im Betrieb} \\ 
   & \hyperref[var:D1bb:1:3]{D1bb\_1\_3} & \textbf{Anteil von Minijobs bis 450 € im Betrieb} \\ 
   & \hyperref[var:D1b:4]{D1b\_4} & \textbf{Vorhandensein von befristet Beschäftigten im Betrieb} \\ 
   & \hyperref[var:D1b:1:4]{D1b\_1\_4} & \textbf{Anzahl von befristeten Beschäftigten im Betrieb} \\ 
   & \hyperref[var:D1bb:1:4]{D1bb\_1\_4} & \textbf{Anteil von befristeten Beschäftigten im Betrieb} \\ 
   & \hyperref[var:D1b:5]{D1b\_5} & \textbf{Vorhandensein von Praktikanten im Betrieb} \\ 
   & \hyperref[var:D1b:1:5]{D1b\_1\_5} & \textbf{Anzahl von Praktikanten im Betrieb} \\ 
   & \hyperref[var:D1bb:1:5]{D1bb\_1\_5} & \textbf{Anteil von Praktikanten im Betrieb} \\ 
   & \hyperref[var:D1b:7]{D1b\_7} & \textbf{Vorhandensein von Auszubildenden im Betrieb} \\ 
   & \hyperref[var:D1b:1:7]{D1b\_1\_7} & \textbf{Anzahl von Auszubildenden im Betrieb} \\ 
   & \hyperref[var:D1bb:1:7]{D1bb\_1\_7} & \textbf{Anteil von Auszubildenden im Betrieb} \\ 
   & \hyperref[var:D1c:1]{D1c\_1} & \textbf{Vorhandensein von freien Mitarbeitern im Betrieb} \\ 
   & \hyperref[var:D1c:1:1]{D1c\_1\_1} & \textbf{Anzahl von freien Mitarbeitern im Betrieb} \\ 
   & \hyperref[var:D1cc:1:1]{D1cc\_1\_1} & \textbf{Anteil von freien Mitarbeitern im Betrieb} \\ 
   & \hyperref[var:D1c:2]{D1c\_2} & \textbf{Vorhandensein von Beschäftigten von Werkvertragsunternehmen im Betrieb} \\ 
   & \hyperref[var:D1c:1:2]{D1c\_1\_2} & \textbf{Anzahl von Beschäftigten von Werkvertragsunternehmen im Betrieb} \\ 
   & \hyperref[var:D1cc:1:2]{D1cc\_1\_2} & \textbf{Anteil von Beschäftigten von Werkvertragsunternehmen im Betrieb} \\ 
   & \hyperref[var:D1c:3]{D1c\_3} & \textbf{Vorhandensein von Leiharbeitskräften im Betrieb} \\ 
   & \hyperref[var:D1c:1:3]{D1c\_1\_3} & \textbf{Anzahl von Leiharbeitskräften im Betrieb} \\ 
   & \hyperref[var:D1cc:1:3]{D1cc\_1\_3} & \textbf{Anteil von Leiharbeitskräften im Betrieb} \\ 
   & \hyperref[var:D1d:1]{D1d\_1} & \textbf{Vorhandensein von Beschäftigten mit Hochschulabschluss im Betrieb} \\ 
   & \hyperref[var:D1d:1:1]{D1d\_1\_1} & \textbf{Anzahl von Beschäftigten mit Hochschulabschluss im Betrieb} \\ 
   & \hyperref[var:D1dd:1:1]{D1dd\_1\_1} & \textbf{Anteil von Beschäftigten mit Hochschulabschluss im Betrieb} \\ 
   & \hyperref[var:D1d:2]{D1d\_2} & \textbf{Vorhandensein von Beschäftigten mit Berufsausbildung/ohne Hochschulabschluss} \\ 
   & \hyperref[var:D1d:1:2]{D1d\_1\_2} & \textbf{Anzahl von Beschäftigten mit Berufsausbildung/ohne Hochschulabschluss} \\ 
   & \hyperref[var:D1dd:1:2]{D1dd\_1\_2} & \textbf{Anteil von Beschäftigten mit Berufsausbildung/ohne Hochschulabschluss} \\ 
   & \hyperref[var:D1d:3]{D1d\_3} & \textbf{Vorhandensein von un- und angelernten Beschäftigten im Betrieb} \\ 
   & \hyperref[var:D1d:1:3]{D1d\_1\_3} & \textbf{Anzahl von un- und angelernten Beschäftigten im Betrieb} \\ 
   & \hyperref[var:D1dd:1:3]{D1dd\_1\_3} & \textbf{Anteil von un- und angelernten Beschäftigten im Betrieb} \\ 
   & \hyperref[var:D1e]{D1e} & \textbf{Anzahl der Beschäftigten unter 30 Jahre im Betrieb} \\ 
   & \hyperref[var:D1ee:1]{D1ee\_1} & \textbf{Anteil der Beschäftigten unter 30 Jahre im Betrieb} \\ 
   & \hyperref[var:D1f]{D1f} & \textbf{Anzahl der Beschäftigten über 55 Jahre im Betrieb} \\ 
   & \hyperref[var:D1ff:1]{D1ff\_1} & \textbf{Anteil der Beschäftigten über 55 Jahre im Betrieb} \\ 
   & \hyperref[var:D1g]{D1g} & \textbf{Anzahl der Beschäftigten mit Migrationshintergrund im Betrieb} \\ 
   & \hyperref[var:D1gg:1]{D1gg\_1} & \textbf{Anteil der Beschäftigten mit Migrationshintergrund im Betrieb} \\ 
   & \hyperref[var:F3]{F3} & \textbf{Anzahl der darin einbezogenen Beschäftigten} \\ 
   & \hyperref[var:F3:1]{F3\_1} & \textbf{Anteil der darin einbezogenen Beschäftigten} \\ 
   & \hyperref[var:M4]{M4} & \textbf{Anzahl der Frauen unter den Betriebsratsmitgliedern} \\ 
   & \hyperref[var:M5a]{M5a} & \textbf{Anzahl der Betriebsratsmitglieder mit der Mitgliedschaft in einer Gewerkschaft} \\ 
   & \hyperref[var:M5b]{M5b} & \textbf{Anzahl der Betriebsratsmitglieder m. d. Mitgliedschaft in einer DGB-Gewerkschaft} \\ 
   & \hyperref[var:M6]{M6} & \textbf{Anzahl der Betriebsratsmitglieder mit einem Hochschulabschluss} \\ 
   & \hyperref[var:M6a]{M6a} & \textbf{Anzahl der Betriebsratsmitglieder mit einer abgeschlossenen Berufsausbildung} \\ 
   & \hyperref[var:M6b]{M6b} & \textbf{Anzahl der teilzeitbeschäftigen Betriebsratsmitglieder} \\ 
   & \hyperref[var:M6c]{M6c} & \textbf{Anzahl der befristet beschäftigten Betriebsratsmitglieder} \\ 
   & \hyperref[var:M6e]{M6e} & \textbf{Anzahl der Betriebsratsmitglieder mit Migrationshintergrund} \\ 
   & \hyperref[var:M6f]{M6f} & \textbf{Anzahl der Betriebsratsmitglieder unter 30 Jahre} \\ 
   & \hyperref[var:M6g]{M6g} & \textbf{Anzahl der Betriebsratsmitglieder über 55 Jahre} \\ 
   & \hyperref[var:M7]{M7} & \textbf{Anzahl der voll freigestellten Betriebsratsmitglieder} \\ 
   & \hyperref[var:M8]{M8} & \textbf{Anzahl der teilweise freigestellten Betriebsratsmitglieder} \\ 
   & \hyperref[var:M11]{M11} & \textbf{Anzahl der Betriebsratsmitglieder in der ersten Amtsperiode} \\ 
   & \hyperref[var:M11a]{M11a} & \textbf{Anzahl der Betriebsratsmitglieder in der zweiten Amtsperiode} \\ 
   & \hyperref[var:M11b]{M11b} & \textbf{Anzahl der Betriebsratsmitglieder in der dritten oder einer höheren Amtsperiode} \\ 
   \midrule
\multirow{2}{2.5cm}{\textbf{Angaben vergröbert}} & \hyperref[var:D1]{D1} & \textbf{Anzahl der Beschäftigten im Betrieb} \\ 
   & \hyperref[var:I7]{I7} & \textbf{Höhe des Krankenstandes am 01.12.2014 in Prozent} \\ 
   & \hyperref[var:M1]{M1} & \textbf{Existenzdauer des Betribsrats in Jahren} \\ 
   & \hyperref[var:M2]{M2} & \textbf{Wahljahr des Betriebsrats} \\ 
   & \hyperref[var:M3]{M3} & \textbf{Anzahl der Betriebsratsmitglieder} \\ 
   & \hyperref[var:M13]{M13} & \textbf{Alter des Betriebsratsvorsitzenden} \\ 
   & \hyperref[var:R1a]{R1a} & \textbf{Anzahl der Beschäftigten im Gesamtunternehmen} \\ 
  
	
\end{longtable}

\subsection{Ergänzte Variablen}\label{var_ergaenzt}

Dem Scientific Use File wurden vier Gewichtungsfaktoren sowie eine Interviewerkennung hinzugefügt. Die Herleitung der Gewichtungsfaktoren entnehmen Sie bitte dem Methodenbericht der WSI-Betriebsrätebefragung 2015. Dabei sollten Sie insbesondere ber?cksichtigen, dass Sie, je nachdem, ob Sie Fragen der Kurz- oder Langversion des Fragebogens auswerten, unterschiedliche Gewichtungsfaktoren verwenden müssen. Zudem ist zu beachten, dass den Gewichtungsfaktoren unterschiedliche Grundgesamtheiten zugrunde liegen.\footnote{Das WSI empfiehlt die Verwendung der Gewichtungsfaktoren für die Grundgesamtheit der Betriebe mit Betriebsrat.} Mithilfe der Interviewerkennung können Clusterungseffekte z.B. in Mehrebenenanalysen berücksichtigt werden. Die ergänzten Variablen und ihre Labels lauten:

\begin{enumerate}

\item \hyperref[var:suf:gewab:k]{gewab\_k}: Gewichtungsfaktor Grundgesamtheit aller Betriebe (Kurzversion)
\item \hyperref[var:suf:gewab:l]{gewab\_l}: Gewichtungsfaktor Grundgesamtheit aller Betriebe (Langversion)
\item \hyperref[var:suf:gewbr:k]{gewbr\_k}: Gewichtungsfaktor Grundgesamtheit Betriebe mit Betriebsrat (Kurzversion)
\item \hyperref[var:suf:gewbr:l]{gewbr\_l}: Gewichtungsfaktor Grundgesamtheit Betriebe mit Betriebsrat (Langversion)
\item \hyperref[var:suf:internr:n]{internr\_n}: Interviewernummer
\item \hyperref[var:suf:westost]{westost}: WESTOST
\item \hyperref[var:suf:branche10]{branche10}: Branche (WZ 2008)

\end{enumerate}
